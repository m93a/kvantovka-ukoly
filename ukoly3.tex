% !TEX program = xelatex

\documentclass[10pt,a4paper]{article}
\usepackage[top = 1.5cm, bottom = 1.5cm, left = 1.5cm, right = 1.5cm]{geometry}

\usepackage{titling}
\usepackage[czech]{babel}
\usepackage{graphicx}
\usepackage{lmodern}
\usepackage{hyperref}
\usepackage{setspace}
\usepackage{csvsimple}

\usepackage{amsmath}
\usepackage{amssymb}
\usepackage{gensymb}
\usepackage{mathtools}
\usepackage{units}
\usepackage{bm}
\delimitershortfall=-1pt

\usepackage{gnuplottex}
\usepackage{epstopdf}

% no page break
\newenvironment{absolutelynopagebreak}
  {\par\nobreak\vfil\penalty0\vfilneg
   \vtop\bgroup}
  {\par\xdef\tpd{\the\prevdepth}\egroup
   \prevdepth=\tpd}


% redefine \sqrt
\usepackage{letltxmacro}
\makeatletter
\let\oldr@@t\r@@t
\def\r@@t#1#2{%
\setbox0=\hbox{$\oldr@@t#1{#2\,}$}\dimen0=\ht0
\advance\dimen0-0.2\ht0
\setbox2=\hbox{\vrule height\ht0 depth -\dimen0}%
{\box0\lower0.4pt\box2}}
\LetLtxMacro{\oldsqrt}{\sqrt}
\renewcommand*{\sqrt}[2][\ ]{\oldsqrt[#1]{#2\,}\,}
\makeatother

\DeclarePairedDelimiter\ceil{\lceil}{\rceil}
\DeclarePairedDelimiter\floor{\lfloor}{\rfloor}

\LetLtxMacro{\oldhbar}{\hbar}
\AtBeginDocument{\renewcommand*{\hbar}{{\mkern-1mu\raisebox{-0.055em}{$\mathchar'26$}\mkern-8mu\mathrm{h}}}}

\def\ph{\phantom}
\def\vph{\vphantom}
\def\hph{\hphantom}
\def\rzw{\mathrlap}
\def\lzw{\mathllap}
\def\czw{\mathclap}

\def\?{\mathit{?}}

\def\+{+\!\!}
\def\-{-\!\!}

\newcommand{\comm}[2]{\left[ #1, #2 \right]}
\newcommand{\const}[1]{\text{#1}}
\newcommand{\norm}[1]{\left\lVert#1\right\rVert}

\newcommand{\mat}[1]{
    \begin{pmatrix}
        #1
    \end{pmatrix}
}

\newcommand{\mata}[2]{
    \left(
    \begin{array}{@{}#1@{}}
        #2
    \end{array}
    \right)
}

\newcommand{\smat}[2][1]{
    \scalebox{#1}{$\mat{#2}$}
}

\newcommand{\tg}{\operatorname{tg}}
\newcommand{\cotg}{\operatorname{cotg}}

\renewcommand{\d}[1]{\;\const{d}#1}
\newcommand{\dd}[2]{\frac{\const{d} #1}{\const{d} #2} \;}
\newcommand{\pd}[2]{\frac{\partial  #1}{\partial  #2} \;}

\newcommand{\bra}[1]{\left< #1 \right|}
\newcommand{\ket}[1]{\left| #1 \right>}
\newcommand{\braket}[2]{\left< #1 \middle| #2 \right>}

\newcommand{\bhat}[1]{\hat{\bm{#1}}}

\newcommand{\e}[1]{\const{e}^{#1}}
\renewcommand{\i}{\const{i}}

\begin{document}

\title{Kvantová mechanika I: Domácí úkoly}
\author{Michal Grňo}
\date{\today}

\maketitle

\section{Cvičení 6. 11.}

\subsection{Zadání}
V $\mathbb{R}^N$ jsou zadány hypersférické souřadnice $q^i = (r, \, \theta_1, \, ... \, \theta_{N-1})$, pro které platí:
\begin{align*}
    x^1 &= r \sin \theta_1 \sin \theta_2 \, ... \sin \theta_{N-2} \cos \theta_{N-1} \\
    x^2 &= r \sin \theta_1 \sin \theta_2 \, ... \sin \theta_{N-2} \sin \theta_{N-1} \\
    x^3 &= r \sin \theta_1 \sin \theta_2 \, ... \cos \theta_{N-2} \\
    \vdots \\
    x^{N-1} &= r \sin \theta_1 \cos \theta_2 \\
    x^N &= r \cos \theta_1
\end{align*}
Hypersférické souřadnice jsou omezeny na následující intervaly:
\begin{align*}
    r &\in \left[ 0, \infty \right) \\
    \theta_{N-1} &\in \left[ 0, 2\pi \right) \\
    \theta_{k<N-1} &\in \left[ 0, \pi \right)
\end{align*}

Vyjádřete elementy a determinant metrického tenzoru $g$. Vyjádřete Laplaceův operátor $\Delta$ a rozložte ho na radiální a centrifugální části. Vyjádřete Schrödingerovu rovnici pro radiální část vlnové funkce $u_{\ell_N}(r)$ pro sféricky symetrický potenciál. Ukažte, že po substituci $u_{\ell_N}(r) = r^\frac{1-N}{2} f_{\ell_N}(r)$ je Schrödingerova rovnice co do tvaru stejná, nezávisle na $N$, liší se pouze silou centrifugálního členu.
\subsection{Řešení}
Aby se nám se souřadnicemi lépe pracovalo, přepíšeme je do tvaru \textit{součinu posloupnosti}:
\begin{align*}
    x^1 &= r \, \left( \prod_{k=1}^{N-2} \sin \theta_k \right) \, \cos \theta_{N-1}
    \\[5pt]
    x^2 &= r \, \left( \prod_{k=1}^{N-1} \sin \theta_k \right)
    \\[5pt]
    x^n &= r \, \left( \prod_{k=1}^{N-n} \sin \theta_k \right) \, \cos \theta_{N-n+1} \;\;\; \text{pro } n>2
\end{align*}
Protože pracujeme na prostoru s pozitivně definitní metrikou, platí:
\begin{align*}
    g_{ij} = \pd{x^a}{q^i} \pd{x^b}{q^j} \delta_{ab}
    = \sum_k \pd{x^k}{q^i} \pd{x^k}{q^j}
\end{align*}
Spočteme si nejprve derivace (v následujícím $n<N, \; 1<k<N$):
\begin{gather*}
    \pd{x^n}{q^1} = \pd{x^n}{r} = \frac{x^n}{r},
    \hspace{4em}
    \pd{x^1}{q^N} = \pd{x^1}{\theta_{N-1}} = - x^2,
    \hspace{4em}
    \pd{x^2}{q^N} = \pd{x^2}{\theta_{N-1}} = x^1,
    \\[5pt]
    n > 2: \;
    \pd{x^n}{q^k} = \begin{cases}
        k>N-n+2: \; 0 \\
        k=N-n+2: \; - x^n \tg \theta_{k-1} \\
        k<N-n+2: \; x^n \cotg \theta_{k-1}
    \end{cases},
    \hspace{2em}
    \pd{x^1}{q^k} = \begin{cases}
        k=N: \; - x^1 \tg \theta_{k-1} \\
        k<N: \; x^1 \cotg \theta_{k-1}
    \end{cases},
    \hspace{2em}
    \pd{x^2}{q^k} = x^2 \cotg \theta_{k-1}.
\end{gather*}
Pokračujeme výpočtem samotných složek metriky:
\begin{align*}
    g_{11} = \sum_{k=1}^N \pd{x^k}{r} \pd{x^k}{r} = \sum_{k=1}^N \left(\frac{x^k}{r}\right)^2 = \frac{r^2}{r^2} = 1.
\end{align*}
\begin{align*}
    g_{NN} = \sum_{k=1}^N \pd{x^k}{\theta_{N-1}} \pd{x^k}{\theta_{N-1}} = \sum_{k=1}^N \left( \pd{x^k}{\theta_{N-1}} \right)^2 = \left( x^1 \right)^2 + \left( x^2 \right)^2 = r^2 \prod_{k=1}^{N-2} \sin^2 \theta_k.
\end{align*}
\begin{align*}
    g_{1N} = \sum_{k=1}^N \pd{x^k}{r} \pd{x^k}{\theta_{N-1}} = \frac{x^1}{r} \left( -x^2 \right) + \frac{x^2}{r} x^1 = 0.
\end{align*}
\begin{align*}
    g_{nn} &= \sum_{k=1}^N \pd{x^k}{q^n} \pd{x^k}{q^n} = \!\!\! \sum_{k=1}^{N-n+1} \!\! \left( \pd{x^k}{\theta_{n-1}} \!\!\right)^2 \!\! + \left(\pd{x^{N-n+2}}{\theta_{n-1}}\!\!\right)^2 = \!\!\! \sum_{k=1}^{N-n+1} \!\! \left( x^k \cotg \theta_{n-1} \right)^2 \!\! + \left( x^{N-n+2} \tg \theta_{n-1} \right)^2 \\[2pt]
    &= \left(x^1\right)^2 + \left(x^2\right)^2 + \sum_{k=2}^{N-n+1} \left( r \cotg \theta_{n-1} \cos \theta_{N-k+1} \prod_{j=1}^{N-k} \sin \theta_j \right)^2 + \left( x^{N-n+2} \tg \theta_{n-1} \right)^2 \\[2pt]
    &= r^2 \prod_{k=1}^{N-2} \sin^2 \theta_k + r^2 \cotg^2 \theta_{n-1} \sum_{k=2}^{n-2} \cos^2 \theta_{N-k+1} \prod_{j=1}^{N-k} \sin^2 \theta_j + r^2 \tg^2 \theta_{n-1} \cos^2 \theta_{n-1} \prod_{j=1}^{n-2} \sin^2 \theta_j
\end{align*}
\begin{align*}
    g_{ab} = \sum_{k=1}^N \pd{x^k}{q^a} \pd{x^k}{q^b}
    = \pd{x^1}{q^a}\pd{x^1}{q^b} + \pd{x^2}{q^a}\pd{x^2}{q^b} + \sum_{k=3}^N \pd{x^k}{q^a} \pd{x^k}{q^b},
\end{align*}
\begin{align*}
    \det g = r^{2(N-1)} \; \prod_{k=2}^{N-1} \sin^{2(N - k)} \theta_{k-1}
\end{align*}


\end{document}